%!TEX TS-program = xelatex
%!TEX encoding = UTF-8 Unicode
% Awesome CV LaTeX Template for CV/Resume
% This template has been downloaded from:
% https://github.com/posquit0/Awesome-CV
% Author:
% Claud D. Park <posquit0.bj@gmail.com>
% http://www.posquit0.com
% Template license:
% CC BY-SA 4.0 (https://creativecommons.org/licenses/by-sa/4.0/)
%
% Comentários feitos por Marcos Muniz [telegram: @munizm]
%
%-------------------------------------------------------------------------------
% CONFIGURAÇÕES
%-------------------------------------------------------------------------------
% papel A4 "a4paper" definido por padrão , use "letterpaper" para modelo de carta EUA.
\documentclass[12pt, a4paper]{awesome-cv}

% Configura margens das páginas com o comando "\geometry{}"
\geometry{left=1.4cm, top=.8cm, right=1.4cm, bottom=1.8cm, footskip=.5cm}

\usepackage[portuguese]{babel} % Define pacote de idioma para portugues. Não altere.

% Especifica o diretório onde as fontes estão incluídas
\fontdir[fonts/]

% Cores dos destaques
% Awesome Colors: awesome-emerald, awesome-skyblue, awesome-red, awesome-pink, awesome-orange
%                 awesome-nephritis, awesome-concrete, awesome-darknight

% Com este comando você pode escolher as cores descritas acimas.
% Altere somente o segundo bloco.
%\colorlet{awesome}{awesome-red} % ESTE ESTÁ DESABILITADO

% Com este comando você define uma cor personalizada Exemplo: 000000 = preto; ff0000 = vermelho
 \definecolor{awesome}{HTML}{000000} % ESTE ESTÁ HABILITADO
 
% ATENÇÃO, PARA OS COMANDOS DE CORES, DEVE-SE DEIXAR SOMENTE 1 ATIVO.
% PARA REMOVA O "%" PARA O QUAL VOCÊ DESEJA HABILITAR
% ADICIONE "%" PARA O QUE VOCÊ DESEJA DESABILITAR

% Cores do texto, não acho necessário fazer alterações aqui.
% Descomente caso você queira especificar alguma cor.
% \definecolor{darktext}{HTML}{414141}
 \definecolor{text}{HTML}{000000}%333333
 \definecolor{graytext}{HTML}{000000}%5D5D5D
 \definecolor{lighttext}{HTML}{333333}%999999

% Set false if you don't want to highlight section with awesome color
% Deixe "false" se você não quer destacar a seção com "awesome color" / Utilize "true" para habilitar o destaque de cores.
\setbool{acvSectionColorHighlight}{false}


% Caso você deseje alterar o separador de informação social com o símbulo "|" para algum outro
% Não vejo necessidade de alterar
\renewcommand{\acvHeaderSocialSep}{\quad\textbar\quad}

\makeatletter % Não altere
\patchcmd{\@sectioncolor}{\color}{\mdseries\color}{}{} % Não altere
\makeatother % Não altere

%-------------------------------------------------------------------------------
%	INFORMAÇÕES PESSOAIS
%	Comente quaisquer linhas abaixo que achar desnecessária
%-------------------------------------------------------------------------------
% Opções disponíveis: circle|rectangle,edge|noedge,left|right
% Para remover a foto, comente a linha abaixo
\photo[circle,noedge,right]{profile}
\name{Nome}{Sobrenome}
\position{Cargo pretendido ou título profissional}
\address{Seu Endereço, nº - Bairro - Cidade/UF}

\mobile{(55 XX) XXXXX-XXXX}
\email{seu@email.com}
\homepage{www.seu-site.com}
\github{github-id}
\linkedin{linkedin-id}
\gitlab{gitlab-id}
\stackoverflow{SO-id}{SO-name}
\twitter{@twit}
\skype{skype-id}
\reddit{reddit-id}
\extrainfo{Outras informações, telefone fixo, etc}

% REPETINDO, CASO NÃO QUEIRA ALGUMA LINHA ACIMA, COMENTE COM UM "%"

\quote{``Uma citação que goste ou ache inspiradora." - Autor da Citação}


%-------------------------------------------------------------------------------
\begin{document}

% Imprime o cabeçalho com as informações acima
% Códigos de alinhamento (C: centro, L: esquerda, R: direita)
\makecvheader[C]

% Imprime o rodapé.
\makecvfooter
  {\today} % Insere a data "hoje", dia em que a compilação é feita.
  {Curriculim Vit\ae}
  {\thepage}


%-------------------------------------------------------------------------------
%	CONTEÚDO DO CURRÍCULO
%   Cada seção é importada separadamente, abra cada arquivo para modificar o conteúdo.
%-------------------------------------------------------------------------------
%-------------------------------------------------------------------------------
%	SECTION TITLE
%-------------------------------------------------------------------------------
\cvsection{Resumo}


%-------------------------------------------------------------------------------
%	CONTENT
%-------------------------------------------------------------------------------
\begin{cvparagraph}

%---------------------------------------------------------
Um resumo de suas principais atribuições, etc e tal. Seja criativo, mas não abuse deste campo.
\end{cvparagraph}

%-------------------------------------------------------------------------------
%	TÍTULO
%-------------------------------------------------------------------------------
\cvsection{Formação Acadêmica}


%-------------------------------------------------------------------------------
%	CONTEÚDO
%-------------------------------------------------------------------------------
\begin{cventries}

%---------------------------------------------------------
  \cventry
    {Engenharia} % Formação
    {Universidade ali da esquina} % Instituição
    {Cidade, UF} % Local
    {20010 - 2019} % Período
    {
      \begin{cvitems} % Descrição
      	% Caso queira remover a descrição, apague todas estas linhas
      	% de \begin{cvitems} até \end{cvitems}
        \item {Descrição 1}
        \item {Descrição 2}
      \end{cvitems}
    }

%---------------------------------------------------------

% Para adicionar outra formação, apenas descomente as linhas abixo
% E preencha com as informações requeridas
%  \cventry
%	{} % Formação
%	{} % Instituição
%	{} % Local
%	{} % Período
%	{} % Descrição

%---------------------------------------------------------

\end{cventries}

%-------------------------------------------------------------------------------
%	TÍTULO
%-------------------------------------------------------------------------------
\cvsection{Experiência Profissional}


%-------------------------------------------------------------------------------
%	CONTEÚDO
%-------------------------------------------------------------------------------
\begin{cventries}

%---------------------------------------------------------
  \cventry
    {Cargo Ocupado} % Job title
    {Empresa} % Organization
    {Cidade/UF} % Location
    {Jan. 2019 - Dez. 2019} % Date(s)
    {
      \begin{cvitems} % Description(s) of tasks/responsibilities
        \item {Descrição das atividades}
        \item {Descrevendo outra atividade}
        \item {Descrição nº 3}
      \end{cvitems}
    }

%---------------------------------------------------------
\cventry
{Cargo Ocupado} % Job title
{Empresa} % Organization
{Cidade/UF} % Location
{Jan. 2019 - Dez. 2019} % Date(s)
{Exemplo onde a descrição pode ser feita em forma de parágrafo. Caso tenha gostado deste modelo, você pode simplesmente copiar e colar o bloco onde este texto se encontra.}

%---------------------------------------------------------
%\cventry
%{Cargo Ocupado} % Job title
%{Empresa} % Organization
%{Cidade/UF} % Location
%{Jan. 2019 - Dez. 2019} % Date(s)
%{
%	\begin{cvitems} % Description(s) of tasks/responsibilities
%		\item {Descrição das atividades}
%		\item {Descrevendo outra atividade}
%		\item {Descrição nº 3}
%	\end{cvitems}
%}

%-------------------------------------------------------------------------------
%	TÍTULO
%-------------------------------------------------------------------------------
\cvsection{Cursos e Treinamentos}


%-------------------------------------------------------------------------------
%	CONTEÚDO
%-------------------------------------------------------------------------------
\begin{cventries}

%---------------------------------------------------------
  \cventry
    {\textbf{Instituição realizadora do curso/treinamento}}{}{}{}
    {
      \begin{cvitems} 
       \item {Nome do Curso 1}
       \item {Nome do Curso 2}
      \end{cvitems}
    }

%---------------------------------------------------------
\cventry
{\textbf{Outra Instituição realizadora do curso/treinamento}}{}{}{}
{
	\begin{cvitems} 
		\item {Nome do Curso 3}
		\item {Nome do Curso 4}
	\end{cvitems}
}

%---------------------------------------------------------


%-------------------------------------------------------
\end{cventries}

\cvsection{Habilidades}
\begin{cventries}

\cventry{}{}{}{}
    {\begin{cvitems}
    \vspace{-0.5 cm}
        \item \textbf{Língua Estrangeira}: Inglês, Espanhol, Alemão, Russo, Mandarim, Valiriano e Dothraki
        \item \textbf{Software CAD}: AutoCad, SolidWorks, etc...
        \item \textbf{Office}: Pacote Microsoft Office, Pacote LibreOffice
        \item \textbf{Outras habilidades}: Consigo espirrar de olho aberto - Zoeira, não coloque isto, por gentileza.
      \end{cvitems}
      }


 
\end{cventries}

%-------------------------------------------------------------------------------
%	TÍTULO
%-------------------------------------------------------------------------------
\cvsection{Informações Adicionais}


%-------------------------------------------------------------------------------
%	CONTEÚDO
%-------------------------------------------------------------------------------
\begin{cventries}

%---------------------------------------------------------
  \cventry
    {}{}{}{} 
    {
    \vspace{-1 cm}  % recuo de espaçamento vertical de 1 cm
	\begin{cvitems} 
		\item {Disponível para mudanças e viagens}
		\item {Carteira de habilitação AZ}
	\end{cvitems}
    }

\end{cventries}


%-------------------------------------------------------------------------------
\end{document}
